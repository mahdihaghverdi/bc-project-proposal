%   Copyright (c)  2025  Mahdi Haghverdi
%   Permission is granted to copy, distribute and/or modify this document
%    under the terms of the GNU Free Documentation License, Version 1.3
%    or any later version published by the Free Software Foundation;
%    with no Invariant Sections, no Front-Cover Texts, and no Back-Cover Texts.
%    A copy of the license is included in the section entitled "GNU
%    Free Documentation License".

\documentclass[dvipsnames, svgnames, x11names]{article}

\usepackage[paper=a4paper, top=2cm, bottom=4cm, right=4.3cm, left=4.3cm]{geometry}

% URLs and hyperlinks ------------------------------
\usepackage{xcolor}
\usepackage{hyperref}
\hypersetup{
    colorlinks=true,
    linkcolor=DarkBlue,
    filecolor=magenta,
    urlcolor=blue,
}
\usepackage{xurl}
%---------------------------------------------------

\usepackage{enumitem,amssymb}
\newlist{todolist}{itemize}{2}
\setlist[todolist]{label=$\square$}

\usepackage{adjustbox}
\usepackage{graphicx}
\usepackage{xepersian}
\settextfont{Yas}  % Put your desired font here

\renewcommand{\arraystretch}{1.23}  % increase if you want more white space in tables

\begin{document}
\begin{center}
{\small به نام خدا}

\vspace{5mm}
\includegraphics[scale=0.3]{images/logo}

{\small دانشکده‌ی مهندسی کامپیوتر}
\vspace{5mm}

\begin{adjustbox}{width=\textwidth}
\begin{tabular}{|p{\textwidth}|}
\hline
عنوان پروژه (فارسی):‌
% Put Farsi name here
\\
عنوان پروژه (انگلیسی):
\begin{latin}
% Put English name here
\end{latin}
\\ \hline
\end{tabular}
\end{adjustbox}

\vspace{2mm}

\begin{adjustbox}{width=\textwidth}
\begin{tabular}{|c|c|c|c|c|p{2cm}|}
\cline{2-6}
\multicolumn{1}{c|}{} &

نام دانشجو & 
شماره‌ دانشجویی &
بسته اصلی و فرعی &
تعداد واحد گذرانده &
امضا \\
\hline
1 &
% Name
&
% Student numner
&
% Your packages
&
% Your number of courses taken
&
\\
\hline
% Second student if present
% 2 &
% Name
% &
% Student numner
% &
% Your packages
% &
% Your number of courses taken
% &
% \\
% \hline
\end{tabular}
\end{adjustbox}

\vspace{2mm}

\begin{adjustbox}{width=\textwidth}
\begin{tabular}{|p{0.6\textwidth}|p{0.378\textwidth}|}
\hline
استاد راهنمای پروژه: % His/Her name here
&
\\
\cline{1-1}
نظر استاد راهنما:  % nothing needed in the document
\vspace{4cm}
& 
\vspace{3.6cm}
امضا استاد راهنما و تاریخ: % nothing needed in the document
\\
\hline
\end{tabular}
\end{adjustbox}

\vspace{3mm}

\begin{adjustbox}{width=\textwidth}
\begin{tabular}{|p{\textwidth}|}
\hline
این پیشنهاد در تاریخ
\hspace{2cm} % nothing needed in the document
در شورای گروه
\hspace{2cm} % nothing needed in the document
مطرح گردید و
% nothing needed in the document
\begin{todolist}
\item 
بدون  تغییر مورد تصویب قرار گرفت.

\item
با شرایط زیر مورد تصویب قرار گرفت.

\item
به دلایل زیر مورد تصویب قرار نگرفت.
\end{todolist}

\begin{center}
\begin{tabular}{|p{0.9\textwidth}|}
\hline
\vspace{5cm} \\ % nothing needed in the document
\hline
\end{tabular}
\end{center}

نام عضو هیئت علمی بررسی کننده: 
\hspace{4cm}
تاریخ و امضا: % nothing needed in the document
\\
\hline
\end{tabular}
\end{adjustbox}
\end{center}

\newpage
\begin{enumerate}
\item 
موضوع پروژه و اهداف آن را به اختصار شرح دهید.

% Keep one empty line
% {Explain here}

\item 
روش انجام پروژه

% Keep one empty line
% {Explain here}

\item 
آیا این پروژه و یا مشابه آن قبلا انجام شده است؟ اگر بله تفاوت‌های این پروژه با پروژه‌های قبلی را (در صورت وجود) ذکر کنید.

% Keep one empty line
% {Explain here}

\item 
طرح تجاری خود در رابطه با تجاری‌سازی و به سودرسانی پروژه‌ی انجام شده را (در صورت وجود) توضیح دهید.

% Keep one empty line
% {Explain here}

\item 
امکانات موردنیاز جهت انجام پروژه را ذکر کنید.

% Keep one empty line
% {Explain here}

\item 
مهم‌ترین منابع و مراجع لازم برای انجام پروژه پیشنهادی را نام ببرید.

% Keep one empty line
% {Explain here}
% \url{www.google.com}
% \href{www.google.com}{this is an href}

% For citation use:
% \cite{boneh2004short}

\end{enumerate}

%\bibliographystyle{ieeetr-fa}
%\bibliography{ref.bib}

% Use TexStudio
% First F5, then F8 and then F5 to see the results including the biblographies
\end{document}
